\documentclass{article}

% Shell lines
\usepackage{xcolor}
\usepackage{listings}
\lstdefinestyle{BashInputStyle}{
  language=bash,
  basicstyle=\small\sffamily,
  numbers=left,
  numberstyle=\tiny,
  numbersep=3pt,
  frame=tb,
  columns=fullflexible,
  backgroundcolor=\color{yellow!20},
  linewidth=0.9\linewidth,
  xleftmargin=0.1\linewidth
}

\title{Git Cheat Sheet}
\author{Patrick Rabier}

\begin{document}
\maketitle

\section{Basic Git Branching}

This section will help you manipulate branches

\subsection{Displaying existing branches}
\begin{lstlisting}[style=BashInputStyle]
    #  git branch
\end{lstlisting}

\subsection{Creating a branch}
Use the following command to create a branch :
\begin{lstlisting}[style=BashInputStyle]
    # git branch [branch name]
\end{lstlisting}
This only creates the branch. It does not switch to it

\subsection{Switching to a branch}
\begin{lstlisting}[style=BashInputStyle]
    #  git checkout [branch name]
\end{lstlisting}
Modifications you make on a branch won't affect other branches

\subsection{Pushing a branch to a remote repository}
\begin{lstlisting}[style=BashInputStyle]
    # git push [repository] [branch name]
\end{lstlisting}
If the branch doesn't exist on the remote repository, it will be created.
If the branch already exists, it will be updated with the latest commits
Example :
\begin{lstlisting}[style=BashInputStyle]
    # git push origin mybranch
\end{lstlisting}
'origin' is typically the main repository. However, others may be added using the following command :
\begin{lstlisting}[style=BashInputStyle]
    # git remote add [repository name] [repository url]
\end{lstlisting}

\subsection{Deleting a local branch}
\begin{lstlisting}[style=BashInputStyle]
    #  git branch -d [branch name]
\end{lstlisting}
This will only delete the branch locally

\subsection{Deleting a remote branch}
\begin{lstlisting}[style=BashInputStyle]
    #  git push [remote repository name] :[branch name]
\end{lstlisting}
This will only delete the branch on the remote repository
Example :
\begin{lstlisting}[style=BashInputStyle]
    #  git push origin :mybranch
\end{lstlisting}

\end{document}
